\par Dustin E. Stansbury's article "Natural Scene Statistics Account for the Representation of 
Scene Categories in Human Visual Cortex" describes a model by which the human brain 
aggregates information about visual subjects to represent perceived scene categories. Using regression and Latent Dirchlet allocation (LDA) models, the researchers were able to learn and predict voxel responses to different categories of image scenes, and specific objects in the image, known as voxel encoding. Additionally, they attempted to build a reverse  model, using voxel responses to predict the images being viewed in terms of image category and specific objects composing the image. This is described as voxel decoding.

\par The data used in our study was relatable to this format, as the fMRI data was collected from human subjects who were listening to the movie Forrest Gump, providing audio stimuli. We initially had planned to similarly attempt building forward and reverse models for voxel and scene prediction, but an early limitation we came across was that instead of images of scenes, we had audio descriptions of the scenes that were presented to 
human subjects. However, we continued under the justification that the response would still stimulate distinct parts of the brain which can be modeled. These stimulated regions should be correlated and comparable to visual images of the objects, as the concept formulated by the brain remains the same, and the audio description is derived from the video. If audio and visual formulations of concepts are not the same, they at least both are measurable by BOLD response and the method will still be comparable even if the results would not.

\par In addition to testing multiple fMRI preprocessing pipelines, we explored different natural language processing techniques to determine the best way to parse through the scene descriptions. After testing a variety of analysis models, we determined $K$-neareset neighbor classification, ridge and lasso regression modeling and neural network models to be most accurate and informative. LDA, as used in the paper, was not needed to create general scene categories because categories were included in our data, along with a more detailed description. We thus did not explicitly attempt to replicate the paper, but aimed instead to follow their modeling process and style applied to a different dataset.
