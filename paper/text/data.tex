\par \indent 

\subsection{Original Dataset}
\par The dataset used belongs to a study conducted using high-resolution functional magnetic resonance (fMRI) to analyze stimulation due to visual stimulii. 20 participants recorded at high field strength (7 Tesla) during prolonged stimulation with an auditory feature film ("Forrest Gump''). 

\par The movie description was provided in a csv format, with each row containing a 
start and end time (in seconds) corresponding to the time in the movie, along with
a description in German of the events occurring in the movie at that time. 
One noticeable feature of the dataset upon initial observation is that the description
is not continuous-- there exist gaps in the events provided, though minimal.

\par As for the FMRI data, data for 20 subjects were provided, each of whom watched
the movie while being scanned. The FMRI images were taken every two seconds for
the duration of the movie. All of this data was in nii format, and thus compatible
with nibabel libraries in Python.

\par For each of the 20 subjects, answers to pre-experiment survey questions were made available, asking a variety of questions (e.g. left or right handed). 


\subsection{Data Used}
\par The survey data was used to select subject 4 to at least initially build our
model on. She was chosen partially because her fMRI data was one of the most complete
and also for the arbitrary fact that she had perfect pitch.

\par All of the movie description data was used, as it is applicable to the FMRI time-course data for all subjects. 

\par The survey data was not used beyond this point.