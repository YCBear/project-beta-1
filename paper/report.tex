\documentclass[11pt]{article}

\usepackage[margin=0.75in]{geometry}

\title{Modeling of Semantic Representation in the brain using fMRI response}
\author{
  Sinha, Rishi\\
  \texttt{rishizsinha}
  \and
  Mo, Cindy\\
  \texttt{cxmo}
  \and
  Agrawal, Raj\\
  \texttt{raj4}
  \and
  Wang, Yuan (Aria)\\
  \texttt{ariaaay}
  \and
  Dong, Yucheng (Steve)\\
  \texttt{yuchengdong}
}

\bibliographystyle{siam}

\begin{document}
\maketitle

\abstract{You should have a short abstract.}


\section{Introduction}
Dustin E. Stansbur's article "Natural Scene Statistics Account for the Representation of 
Scene Categories in Human Visual Cortex" describes the method by which the human brain 
aggregrate information about subjects to represent scene categories. Using statical learning 
methods, researchers can learn categories of certain objects, and model fMRI brain signals 
when human subjects are viewed images of scenes using the learned categories. In our study, 
instead of images of scenes, we have audio descriptions of the scenes that are presented to 
human subjects. However, we hypothesize that the response is the same- audio would first be 
processed in the  

\section{Data}
 The original audio description was in German, so we first used Google Translate 
 to convert from German to English. Due to grammatical differences, we decided to
 only keep nouns and verbs, and discarded the adjectives and other words including
 stopwords. Princeton University provides a list of stopwords for text preprocessing.
 Stopwords are the most commonly used natural language words in the English, but have 
 very little meaning. Examples of stop words include "and", "to", and "him". We saved 
 the translated text into a CSV file and parsed in Python. 

\section{Methods}
\section{Results}
\section{Discussion}


\bibliography{project}

\end{document}
